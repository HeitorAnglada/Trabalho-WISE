
\documentclass[journal]{IEEEtran}

% *** GRAPHICS RELATED PACKAGES ***
%
\ifCLASSINFOpdf
  % \usepackage[pdftex]{graphicx}
  % declare the path(s) where your graphic files are
  % \graphicspath{{../pdf/}{../jpeg/}}
  % and their extensions so you won't have to specify these with
  % every instance of \includegraphics
  % \DeclareGraphicsExtensions{.pdf,.jpeg,.png}
\else

\fi
\hyphenation{op-tical net-works semi-conduc-tor}


\begin{document}

\title{WISE a Simulation of ....\\ Subtitle}


\author{Heitor~M.~Anglada, Adan~Pedra, Eduardo~Tireilho, and~Jasmine~P.L.~Araújo}



\markboth{Journal of \LaTeX\ Class Files,~Vol.~14, No.~8, August~2015}%
{Shell \MakeLowercase{\textit{et al.}}: Bare Demo of IEEEtran.cls for IEEE Journals}

\maketitle

\begin{abstract}
The abstract goes here.
\end{abstract}

% Note that keywords are not normally used for peerreview papers.
\begin{IEEEkeywords}
IEEE, IEEEtran, journal, \LaTeX, paper, template.
\end{IEEEkeywords}

\IEEEpeerreviewmaketitle



\section{Introduction}
\IEEEPARstart{I}{n} the digital age, network simulation has become an indispensable tool for the development, testing, and management of computer networks. The ability to model and predict the behavior of complex networks in various scenarios is crucial for ensuring efficiency, security, and reliability. In this context, the WISE (Wireless Information System for Emergency Responders) emerges as a specialized network simulator playing a vital role in the planning and execution of emergency and rescue operations.

Developed to provide realistic solutions for communications in critical scenarios, WISE stands out for its accuracy and applicability in life-or-death situations. Network simulation in emergency contexts is not just a matter of technology but also a challenge of adaptability and rapid response to unforeseen circumstances. WISE meets these needs by enabling planners and network operators to simulate and evaluate various communication strategies in controlled environments that closely mirror reality.

\section{Overview of WISE}

\subsection{Development and History}

WISE was conceived as a solution to the complex challenges faced in emergency communication systems. Developed by a collaboration of technologists and emergency response experts, its inception was driven by the need for a robust, adaptable, and realistic simulation environment. The history of WISE is marked by a continuous evolution, shaped by the real-world experiences of first responders and advancements in wireless communication technologies.

\subsection{Purpose and Utilization}

The primary purpose of WISE is to simulate wireless communication networks in emergency scenarios. It allows for the modeling of various network configurations and their behavior under different emergency conditions. This simulation tool is extensively used by emergency planners, network engineers, and first responders to plan and optimize communication strategies for disaster response, search and rescue operations, and other critical situations where communication is vital. The use of WISE extends beyond mere theoretical modeling; it provides actionable insights that can be directly applied to real-life emergency response plans.

\subsection{Technical Architecture}

WISE's technical architecture is a blend of sophistication and user-focused design. It encompasses various modules that simulate different aspects of wireless communication networks, including signal propagation, network traffic, and node mobility. The simulator is designed to be scalable, accommodating a range of scenarios from small-scale incidents to large-scale disasters. Its architecture also supports integration with other simulation tools and real-world data sources, enhancing its realism and applicability.

\subsection{Key Features}

\subsubsection{Realistic Simulation} It offers detailed modeling of wireless communication behaviors, considering factors like terrain, weather, and obstacles.

\subsubsection{Versatility} WISE supports various wireless technologies and standards, making it suitable for a wide range of emergency scenarios.

\subsubsection{User Interaction} The simulator provides an interactive environment where users can modify parameters and instantly see the impact on network performance.

\subsubsection{Analytical Tools} WISE includes analytical tools for performance evaluation, such as network throughput, latency, and connectivity metrics.

\section{Use Cases and Applications of WISE}

\subsection{Practical Examples}
The practical applications of WISE in real-world scenarios underscore its significance and utility. One notable example is its use in disaster simulation exercises. Emergency response teams have utilized WISE to simulate communication networks during natural disasters like earthquakes and floods. In these exercises, WISE enabled planners to assess and optimize wireless communication strategies, ensuring uninterrupted communication amidst the chaos of a disaster. Another example includes its application in urban search and rescue operations, where WISE helped in mapping out communication networks in densely populated areas, identifying potential network breakdowns, and proposing solutions.

\subsection{Benefits of Using WISE}
WISE offers several advantages over other network simulators and traditional methods of emergency communication planning. Firstly, its high degree of realism in simulations allows for a more accurate representation of potential communication challenges in emergency scenarios. This precision is critical in ensuring the reliability of communication strategies under stress. Additionally, WISE's ability to simulate a wide range of wireless technologies and protocols makes it a versatile tool, applicable in various emergency contexts. Its user-friendly interface also facilitates wider adoption and use by professionals with different levels of technical expertise. Furthermore, the insights gained from WISE simulations aid in making informed decisions, potentially saving lives and resources during actual emergency situations.

In conclusion, the diverse use cases and inherent benefits of WISE highlight its crucial role in enhancing the preparedness and effectiveness of emergency communication strategies. Its application in real-world scenarios and its advantages over other simulation methods demonstrate its value as a tool in the arsenal of emergency response planning.

\section{Conclusion}
The conclusion goes here.





% if have a single appendix:
%\appendix[Proof of the Zonklar Equations]
% or
%\appendix  % for no appendix heading
% do not use \section anymore after \appendix, only \section*
% is possibly needed

% use appendices with more than one appendix
% then use \section to start each appendix
% you must declare a \section before using any
% \subsection or using \label (\appendices by itself
% starts a section numbered zero.)
%


\appendices
\section{Proof of the First Zonklar Equation}
Appendix one text goes here.

% you can choose not to have a title for an appendix
% if you want by leaving the argument blank
\section{}
Appendix two text goes here.


% use section* for acknowledgment
\section*{Acknowledgment}


The authors would like to thank...


% Can use something like this to put references on a page
% by themselves when using endfloat and the captionsoff option.
\ifCLASSOPTIONcaptionsoff
  \newpage
\fi



% trigger a \newpage just before the given reference
% number - used to balance the columns on the last page
% adjust value as needed - may need to be readjusted if
% the document is modified later
%\IEEEtriggeratref{8}
% The "triggered" command can be changed if desired:
%\IEEEtriggercmd{\enlargethispage{-5in}}

% references section

% can use a bibliography generated by BibTeX as a .bbl file
% BibTeX documentation can be easily obtained at:
% http://mirror.ctan.org/biblio/bibtex/contrib/doc/
% The IEEEtran BibTeX style support page is at:
% http://www.michaelshell.org/tex/ieeetran/bibtex/
%\bibliographystyle{IEEEtran}
% argument is your BibTeX string definitions and bibliography database(s)
%\bibliography{IEEEabrv,../bib/paper}
%
% <OR> manually copy in the resultant .bbl file
% set second argument of \begin to the number of references
% (used to reserve space for the reference number labels box)
\begin{thebibliography}{1}

\bibitem{IEEEhowto:kopka}
H.~Kopka and P.~W. Daly, \emph{A Guide to \LaTeX}, 3rd~ed.\hskip 1em plus
  0.5em minus 0.4em\relax Harlow, England: Addison-Wesley, 1999.

\end{thebibliography}

% biography section
% 
% If you have an EPS/PDF photo (graphicx package needed) extra braces are
% needed around the contents of the optional argument to biography to prevent
% the LaTeX parser from getting confused when it sees the complicated
% \includegraphics command within an optional argument. (You could create
% your own custom macro containing the \includegraphics command to make things
% simpler here.)
%\begin{IEEEbiography}[{\includegraphics[width=1in,height=1.25in,clip,keepaspectratio]{mshell}}]{Michael Shell}
% or if you just want to reserve a space for a photo:

\begin{IEEEbiography}{Michael Shell}
Biography text here.
\end{IEEEbiography}

% if you will not have a photo at all:
\begin{IEEEbiographynophoto}{John Doe}
Biography text here.
\end{IEEEbiographynophoto}

% insert where needed to balance the two columns on the last page with
% biographies
%\newpage

\begin{IEEEbiographynophoto}{Jane Doe}
Biography text here.
\end{IEEEbiographynophoto}

% You can push biographies down or up by placing
% a \vfill before or after them. The appropriate
% use of \vfill depends on what kind of text is
% on the last page and whether or not the columns
% are being equalized.

%\vfill

% Can be used to pull up biographies so that the bottom of the last one
% is flush with the other column.
%\enlargethispage{-5in}



% that's all folks
\end{document}


